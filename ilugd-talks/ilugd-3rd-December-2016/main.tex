% Copyright 2004 by Till Tantau <tantau@users.sourceforge.net>.
%
% In principle, this file can be redistributed and/or modified under
% the terms of the GNU Public License, version 2.
%
% However, this file is supposed to be a template to be modified
% for your own needs. For this reason, if you use this file as a
% template and not specifically distribute it as part of a another
% package/program, I grant the extra permission to freely copy and
% modify this file as you see fit and even to delete this copyright
% notice. 

\documentclass{beamer}

% There are many different themes available for Beamer. A comprehensive
% list with examples is given here:
% http://deic.uab.es/~iblanes/beamer_gallery/index_by_theme.html
% You can uncomment the themes below if you would like to use a different
% one:
%\usetheme{AnnArbor}
%\usetheme{Antibes}
%\usetheme{Bergen}
%\usetheme{Berkeley}
%\usetheme{Berlin}
%\usetheme{Boadilla}
%\usetheme{boxes}
%\usetheme{CambridgeUS}
%\usetheme{Copenhagen}
%\usetheme{Darmstadt}
%\usetheme{default}
%\usetheme{Frankfurt}
%\usetheme{Goettingen}
%\usetheme{Hannover}
%\usetheme{Ilmenau}
%\usetheme{JuanLesPins}
%\usetheme{Luebeck}
%\usetheme{Madrid}
\usetheme{Malmoe}
%\usetheme{Marburg}
%\usetheme{Montpellier}
%\usetheme{PaloAlto}
%\usetheme{Pittsburgh}
%\usetheme{Rochester}
%\usetheme{Singapore}
%\usetheme{Szeged}
%\usetheme{Warsaw}

\title{Linux File Operations and CLI }

%\pgfdeclareimage[height=2.3cm]{cerc-logo}{cerc.png}
\pgfdeclareimage[height=1.5cm]{university-logo}{logo.jpg}
% A subtitle is optional and this may be deleted
%\subtitle{emacs vs vim}

\author{Shyam Saini aka mysticTot } 

\institute[ASET] % (optional, but mostly needed)
{
  Amity School of Engineering and Technology -- {\em (ASET)} \\
  Department of Computer Science\\
  Amity University
  \center
}

\date{ILUG-D \& LinuxChix 2016}

\subject{Linux File Operations}

% If you have a file called "university-logo-filename.xxx", where xxx
% is a graphic format that can be processed by latex or pdflatex,
% resp., then you can add a logo as follows:

\logo{\pgfuseimage{university-logo}}

% Delete this, if you do not want the table of contents to pop up at
% the beginning of each subsection:
\AtBeginSubsection[]
{
	\begin{frame}<beamer>{Agenda}
		\tableofcontents[currentsection,currentsubsection]
	\end{frame}
}

% Let's get started
\begin{document}

\begin{frame}
	\titlepage
\end{frame}

\begin{frame}{Agenda}
	\tableofcontents
	% You might wish to add the option [pausesections]
\end{frame}

% Section and subsections will appear in the presentation overview
% and table of contents.
\section{Linux File Operations}

\subsection{Files Types in Linux}

\begin{frame}{everything is a \textbf{file*}} {File Types in Linux}
%	\textbf{File Types in Linux}\\ 
	{
	\begin{itemize}
	

    \item {
				\textbf{Regular file(-)}
		}
		\end{itemize}
		
    \begin{itemize}
    
	\item{
				\textbf{Directory files(d)}
		}
	\end{itemize}
	
	\begin{itemize}
	\item{
                \textbf{Special files}
                
                \begin{itemize}
		        \item {
				\textbf{Block file(b)}: eg: /dev/sda*, /dev/sdb* files
				\pause
				}
	
				\item {
				\textbf{Character device file(c)}: eg: /dev/tty* files 
				\pause
				}
				
				\item {
				\textbf{Named pipe file or just a pipe file(p)}
				\pause
				}
				
				\item {
				\textbf{Symbolic link file(l)}
				\pause
				}
				
				\item {
				\textbf{Socket file(s)}: eg: /var/run/cups/cups.sock
				}
				
				\end{itemize}
                
                
                
                
                        
            }
    \end{itemize}
    }
    
    \end{frame}
    
    \begin{frame}{everything is a \textbf{file*}}
        \begin{center}
	    \huge{We will focus on Regular files and Directory files}
	    \end{center}
	\end{frame}
   

\subsection{File Operations}

% You can reveal the parts of a slide one at a time
% with the \pause command:
\begin{frame}{File operations}{ on Regular files}


	\begin{itemize}
		\item {
				\textbf{touch}: change file timestamps
				\pause
				}
			\item {   
				\textbf{cat}: concatenate files and print on the standard output
				\pause
			}
			% You can also specify when the content should appear
			% by using <n->:
			%  \item<3-> {
		\item {
				\textbf{cp}: copy files and directories
				\pause
			}
	
		\item{
		        \textbf{rm}: remove files or directories
                \pause
		  }
		  
		  \item{
		        \textbf{more}: file perusal filter for crt viewing  
		        \pause
		  }
		  
		   \item{
		        \textbf{file}: determine file type
		        \pause
		  } 
		  
		  \item{
		        \textbf{find}: search for files in a directory hierarchy 
		        \pause
		  } 
		  
		  \item{
		        \textbf{diff}: compare files line by line 
	            \pause
		  } 
		  
		  \item{
		        \textbf{grep}: print lines matching a pattern  
		        \pause
		  }
		  
	%	  \item{
	%	        \textbf{}:  
	%	  } 
	%	  
	%	  \item{
	%	        \textbf{}:  
	%	  } \item{
	%	        \textbf{}:  
	%	  } \item{
	%	        \textbf{}:  
	%	  }
		  
	\end{itemize}
    \end{frame}
    
    
    \begin{frame}{File operations} {on Directories  or Directory file type}
    

	\begin{itemize}
	
	 \item{
		        \textbf{mkdir}: make directories  
		        \pause
		  }
	
	\item{
		        \textbf{rmdir}: remove empty directories  
		        \pause
		  }
	\item{
		        \textbf{cd}: change directories
		        \pause
		  }
	
	\item{
		        \textbf{rm}: with -r flags it deletes directory   
		        \pause
		  }
	\item{
		        \textbf{cp}: with -r flags it copies the directory content	\pause
		  }
	\item{
		        \textbf{ls}: list files in directory
		        \pause
		  }	  
	\item{
		        \textbf{pwd}: print name of current/working directory
		  }
		  
		  
	
	\end{itemize}
	
\end{frame}

\subsection{File Permissions}

	\begin{figure}[htp]
    \centering
    \includegraphics[width=9cm, height=9cm]{hulk-vs-Linux.png}
    %\caption{Even Hulk is Denied, if not have permissions}
    %\label{fig:lion}
    \end{figure}
    
\begin{frame}{File Permissions}
	\textbf{Permission Groups}

	
	\begin{itemize}

			\item {
					\textbf{user (u)}: apply only the owner of the file or directory
					\pause
					}
			\item {
					\textbf{group (g)}: apply only to the group that has been assigned to the file or directory
					\pause
					}
			\item {
					\textbf{all users (a or o) }:  apply to all other users on the system
					\pause
			}

	\end{itemize}
	
    Permissions Types
	\begin{itemize}

	    \item {
	        \textbf{read (r or 4)}: assign read permissions
	        \pause
	        }
	    
	    \item {
	        \textbf{write (w or 2)}: assign write permissions
	        \pause
	        }
	        
	   \item {
	        \textbf{execute (x or 1 )}: assign execute permissions
	        \pause
	        }
	        
	\end{itemize}
	
\end{frame}
\begin{frame}{File Permissions Commands}
    
	\alert{ Protect Files from Hulk} 
	
	\begin{itemize}
	
	    	\item {
				\textbf{chown}: change file owner and group
				\pause
				}
				
		\item {
				\textbf{chmod}: change file mode bits
				\pause
				}
	
	
	\end{itemize}
	
\end{frame}

\section{OFF Topic!}


\subsection{A Cool Tool to make everyone happy!}
\begin{frame}
\begin{center}
\huge Kexec
\end{center}

\end{frame}


    

    
    \begin{center}
    \includegraphics[width=5cm]{smily.jpeg}
    %\caption{I wonder, How does it works}
    \end{center}
   



\begin{frame}{ Kexec } {analogous to \textbf{exec} command}    
	\textbf{directly boot into a new kernel }
	\pause
	\begin{itemize}
		\item {
			\textbf{kexec -l}: Load the specified kernel into the current kernel
			\pause
			}

		\item {   
			\textbf{kexec -e}: Run the currently loaded kernel

			}

	\end{itemize}

\end{frame}

\begin{frame}
\begin{center}
	\huge{\alert{It seems my system is freezed, But Wait!} }
\end{center}
\end{frame}	

\begin{frame}
	\begin{figure}[htp]
    \centering
    \includegraphics[width=1cm]{happ.jpeg}
    %\caption{Me thinks, How does it works}
    % \label{fig:lion}
    \end{figure}
	\textbf{Aha!, we have booted into new kernel on the fly}

\end{frame}

\begin{frame}{Thats all!}
\begin{center}
\Huge Thank You and Happy Hacking!
\end{center}
	
\end{frame}

\end{document}


